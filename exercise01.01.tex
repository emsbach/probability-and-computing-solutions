\paragraph{Exercise 1.1} We flip a fair coin ten times.
\begin{enumerate}
  \item[(a)] Our sample space $\Omega$ is the set of all possible outcomes of ten
    fair coin flips. All simple events have equal probability. The sample space has
    $2^{10}$ simple events, so each of them has probability $2^{-10}$. We are
    interested in the probability of the event $E_a$ that the number of heads and
    the number of tails are equal; that is, in the event that there are 5 heads.
    There are $\binom{10}{5}$ simple events that fulfil this condition. Consequently,
    \[
      \pr(E_a) = \binom{10}{5} \cdot 2^{-10} = 252 \cdot 2^{-10} = \frac{63}{256}.
    \]

  \item[(b)] We are interested in the probability of the event $E_b$ that there are
    more heads than tails. Let $E_i$ be the event that the number of heads equals
    $i \in [0, 10]$. Then $E = \bigcup_{i = 6}^{10} E_i$. Since $E_1, E_2,..., E_{10}$
    are mutually disjoint,
    \begin{align*}
      \pr\left(E_b\right)
        &= \pr\left(\bigcup_{i = 6}^{10} E_i \right)      = \sum_{i=6}^{10} \pr\left(E_i\right) = 2^{-10} \sum_{i=6}^{10} \binom{10}{i} \\
        &= 2^{-10} \left( 210 + 120 + 45 + 10 + 1\right)  = \frac{193}{512}.
    \end{align*}

  \item[(c)] We are interested in the probability of the event $E_c$ that the $i$th
    flip and the $(11 - i)$th flip are the same for $i \in [1,5]$. Let
    $X_i \in$ \{H, T\} be the outcome of the $i$th flip. Since we are tossing a
    fair coin, $\pr\left(X_i = \text{H} \right) = \pr\left(X_i = \text{T} \right) = \frac{1}{2}$.
    For $X_{11-i}$ holds, $\pr\left(X_{11-i} = X_i \right) = \pr\left(X_{11-i}
    \not= X_i \right) = \frac{1}{2}$. Since the outcome of each toss is independet
    of the outcome of the other nine tosses,
    \[
      \pr\left(E_c\right)
        = \pr\left(\bigcap_{i=1}^5 X_i = X_{11-i}\right)
        = 2^{-5}.
    \]

  \item[(d)] We are interested in the probablity of the event $E_d$ that we flip
    at least four consecutive heads. Let $i \in [1,10]$ and let $E_i$ be the event
    that we flip $i$ consecutive heads. Then,
    \[ \pr\left(E_d\right) = \pr\left(\bigcup_{i=4}^{10} E_i\right).\]
    We have to consider, that the events $E_1,..., E_{10}$ are not mutually disjoint.
    For instance, the sequence (H,H,H,H,T,H,H,H,H,H) is element both of the event
    $E_4$ and $E_5$. Let's have a look at the individual events $E_4,..., E_{10}$
    there $x \in$ \{H,T\}:
    \begin{align*}
      \begin{tabular}{lll}
        $E_4$ &=  &\{(H,H,H,H,T,x,x,x,x,x), (T,H,H,H,H,T,x,x,x,x), (x,T,H,H,H,H,T,x,x,x), \\
              &   &\ \,(x,x,T,H,H,H,H,T,x,x), (x,x,x,T,H,H,H,H,T,x), (x,x,x,x,T,H,H,H,H,T), \\
              &   &\ \,(x,x,x,x,x,T,H,H,H,H)\} \\
        $E_5$ &= &\{(H,H,H,H,H,T,x,x,x,x), (T,H,H,H,H,H,T,x,x,x), (x,T,H,H,H,H,H,T,x,x), \\
              &   &\ \,(x,x,T,H,H,H,H,H,T,x), (x,x,x,T,H,H,H,H,H,T), (x,x,x,x,T,H,H,H,H,H)\} \\
        $E_6$ &=  &\{(H,H,H,H,H,H,T,x,x,x), (T,H,H,H,H,H,H,T,x,x), (x,T,H,H,H,H,H,H,T,x), \\
              &   &\ \,(x,x,T,H,H,H,H,H,H,T), (x,x,x,T,H,H,H,H,H,H)\} \\
        $E_7$ &=  &\{(H,H,H,H,H,H,H,T,x,x), (T,H,H,H,H,H,H,H,T,x), (x,T,H,H,H,H,H,H,H,T), \\
              &   &\ \,(x,x,T,H,H,H,H,H,H,H)\} \\
        $E_8$ &=  &\{(H,H,H,H,H,H,H,H,T,x), (T,H,H,H,H,H,H,H,H,T), (x,T,H,H,H,H,H,H,H,H)\} \\
        $E_9$ &=  &\{(H,H,H,H,H,H,H,H,H,T), (T,H,H,H,H,H,H,H,H,H)\} \\
        $E_{10}$ &=  &\{(H,H,H,H,H,H,H,H,H,H)\}.
      \end{tabular}
    \end{align*}
    Now we can determine the cardinality of $E_4, ..., E_{10}$:
    \begin{align*}
      \begin{tabular}{ll}
        $|E_4| = 2 \cdot 2^5 + 5 \cdot 2^4 = 144$   &$|E_8| = 2 \cdot 2^1 + 1 \cdot 2^0 = 5$ \\
        $|E_5| = 2 \cdot 2^4 + 4 \cdot 2^3 = 64$    &$|E_9| = 2 \cdot 2^0 = 2$ \\
        $|E_6| = 2 \cdot 2^3 + 3 \cdot 2^2 = 28$    &$|E_{10}| = 1.$ \\
        $|E_7| = 2 \cdot 2^2 + 2 \cdot 2^1 = 12$    & \\
      \end{tabular}
    \end{align*}
    The events $E_4 \cup E_5, E_6, E_7, E_8, E_9, E_{10}$ are mutually disjoint.
    Therefore,
    \[ \pr\left(E_d\right) = \pr\left(\bigcup_{i=4}^{10} E_i\right)
      = \pr\left(E_4 \cup E_5 \right) + \sum_{i=6}^{10} \pr\left(E_i\right).\]
    It holds that
    \[
      \pr\left(E_4 \cup E_5 \right)
      = \pr\left(E_4\right) + \pr\left(E_5 \right) + \pr\left(E_4 \cap E_5 \right)
      = 2^{-10} \cdot 144 + 2^{-10} \cdot 64 - 2^{-10} \cdot 2
      = \frac{103}{512}.
    \]
    Now, we can compute the probability for event $E_d$:
    \[ \pr\left(E_d\right)
      = \pr\left(E_4 \cup E_5 \right) + \sum_{i=6}^{10} \pr\left(E_i\right) \\
      = \frac{103}{512} + 2^{-10}(28 + 12 + 5 + 1) \\
      = \frac{63}{256}.
    \]
\end{enumerate}
