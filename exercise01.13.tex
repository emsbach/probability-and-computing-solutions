\textbf{Exercise 1.13}:  Let $D$ be the event that a person has the disorder and
$\bar{D}$ the event that the person has not the disorder. Our samples space $\Omega$
comprises the whole population. Obviously $D$ and $\bar{D}$ are mutually disjoint
sets such that $D \cup \bar{D} = \Omega$. Furthermore, let $P$ be the event that
a person is tested positive. We can now apply Bayes' Law to compute the probability
that a person chosen uniformely at random and tested positive has the disorder.
\[
  \pr(D|P)
    = \frac{\pr(P|D) \cdot \pr(D)}{\pr(P|D) \cdot \pr(D) + \pr(P|\bar{D}) \cdot \pr(\bar{D})}
    = \frac{0.999 \cdot 0.02}{0.999 \cdot 0.02 + 0.005 \cdot 0.98}
    = \frac{0.01998}{0.02488}
    = \frac{999}{1244}
    \approx 0.803.
\]
Hence, the probability that a positive tested person has the disorder is
approximately 0.803.
\\[0.5cm]
